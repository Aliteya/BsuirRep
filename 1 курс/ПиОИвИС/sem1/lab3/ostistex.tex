\documentclass{report}

\usepackage[letterpaper,top=1.8cm,bottom=1.8cm,left=1.8cm,right=1.8cm,marginparwidth=1.6cm]{geometry}

\usepackage{caption}
\usepackage{subcaption}
\usepackage{multicol}
\usepackage[english]{babel}
\usepackage[T2A]{fontenc}
\usepackage[utf8]{inputenc}
\usepackage{titlesec}
\usepackage{graphicx}
\usepackage[export]{adjustbox}

\graphicspath{ {./images/} }

\begin{document}



\renewcommand{\thesection}{\Roman{section}}
\titleformat{\section}{\large\centering\sc}{\thesection. }{0cm}{}[]



\setcounter{section}{3}
\setcounter{page}{200}

\begin{figure}[thp]
\addtocounter{figure}{10} 
\end{figure}

\begin{multicols}{2}



\begin{minipage}{.5\textwidth}
\centering
    

    \includegraphics[scale = 0.71,left]{1.png}

\end{minipage}%
 \captionof{figure}{An example of a logical rule that uses the result of
recognizing a user’s emotion}

\vspace{1cm}

\noindent system knows, then we need to respond to this message
with a greeting with a reference by name and ask the
reason for sadness.

\section{HARDWARE ARCHITECTURE}

\par The implementation of the represented models within
the framework of a hybrid system for analyzing the
emotional state requires appropriate support, both from
the software and hardware. Therefore, the issue of creating a hardware architecture of the system that allows
effectively implementing the functional responsibilities
imposed on the system is one of the significant stages
for achieving the overall goal.
\par The developed hardware architecture of the system
should take into account the requirements and features
of the implementation of both the semantic and neural
network components of the system.



\columnbreak

\begin{minipage}{.5\textwidth}
\centering
\begin{center}
    \includegraphics[scale = 0.6]{2.png}
\end{center}

\end{minipage}%
\captionof{figure}{The system hardware architecture}
\vspace{1cm}

\par The hardware requirements, from the point of view
of the semantic part of the system, include the need
to use computing tools with a processor based on the
’x86’ architecture. This is due to the fact that initially
the OSTIS platform as the basis of the software part
of the system was developed for general-purpose CISC
processors. Using this type of processor allows eliminating compatibility problems, simplifying debugging and
testing the system. Therefore, it is necessary to have a
computing device in the system based on this hardware
architecture.
\par On the other hand, to solve computer vision problems,
modern neural network architectures require support for
tensor operations from the hardware platform, which
allows effectively organizing the process of running
trained neural network models on the target device.
However, general-purpose processors are not suitable for
performing such operations with maximum performance.
Therefore, to increase the speed of the system, it is
necessary to include a special coprocessor device in the
hardware architecture, which allows increasing the speed
of the neural network part of the system.
\par Taking into account the listed requirements, we have
developed a hardware architecture of the system, the
block diagram of which is shown in figure 12:
\par The main elements that build up the hardware of the
system are:
\begin{itemize}
    \item 
 single-board computer (SBC) that serves as a central device, on which the OSTIS virtual machine is
run and, accordingly, the interpretation of intelligent
agents and a list of peripheral devices that perform
the functions of input and output of video and
audio information as well as auxiliary devices that
perform the functions of supporting neural network
\end{itemize}



\end{multicols}

\end{document}
